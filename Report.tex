\documentclass[conference]{IEEEtran}

\usepackage[utf8]{inputenc}
\usepackage{graphicx}
\usepackage{amsmath}
\usepackage{amssymb}
\usepackage{booktabs}

\begin{document}

\title{Sequential Specialist MoE Routing Architecture for Complex Queries}

\author{
\IEEEauthorblockN{M.\ G.\ Shree Harsha}
\IEEEauthorblockA{
Dept.\ of Data Science and Artificial Intelligence\\
IIIT Dharwad, Karnataka, India\\
Email: 24bds037@iiitdwd.ac.in}
}

\maketitle

\begin{abstract}
Large language models are usually trained as dense generalists that must
handle mathematics, explanation, coding, and reasoning within a single
network. This is computationally expensive and often inefficient for
multi-stage queries such as ``solve, explain, and code.'' In this paper
I describe a simple routing architecture that coordinates several small
specialist models. A two-stage loss-based router first selects a subset
of relevant experts for the full query and then assigns individual query
segments (e.g., solve, explain, code) to the most suitable specialist.
The selected experts are executed in a specialist pipeline to produce the
final response. Because each model learns only its own domain, training
is much faster than for a single large generalist, while evaluation on a
50--question PhD-level benchmark shows that the specialist system comes
reasonably close to a stronger 3B dense model on challenging math and
computer-science tasks.
\end{abstract}

\begin{IEEEkeywords}
Mixture-of-Experts, routing, language models, multi-domain reasoning,
task decomposition.
\end{IEEEkeywords}

\section{Introduction}
Dense large language models (LLMs) are typically trained as universal
generalists. A single network is expected to solve mathematical
problems, explain them in natural language, and produce correct code,
all with one set of parameters. This design is simple, but it makes
training extremely expensive and offers limited control over how
different skills are used for multi-step tasks.

Many real queries are naturally multi-stage. A typical example is:
\emph{``Solve the problem, explain the reasoning, and then write the
code.''} For such queries, humans intuitively separate roles: a solver,
a teacher, and a programmer. A single dense LLM is forced to internally
simulate this structure without any explicit routing.

In this work I explore a more direct alternative: a \emph{sequential
specialist Mixture-of-Experts (MoE) routing architecture}. Instead of
one large generalist, I maintain several smaller experts:
a math specialist, a code specialist, and a general instruction model.
A lightweight router decides which experts are relevant for a given
query and which expert should handle each segment of the query.

My goals are:
\begin{itemize}
    \item to reduce training time by training each specialist only on
        its own domain;
    \item to maintain competitive performance on difficult math and
        computer science questions;
    \item to provide a modular architecture that can be extended with
        new specialists.
\end{itemize}

\section{Related Work}
Mixture-of-Experts (MoE) architectures route tokens or hidden states to
a subset of experts inside a single large model. These systems achieve
parameter efficiency but typically operate at the token level and do not
explicitly decompose a query into high-level tasks.

Cascaded systems run a small model first and call a larger model only
when necessary. This reduces inference cost but still depends on a
single strong generalist and does not support structured multi-stage
routing.

Multi-agent and ensemble approaches combine several LLMs, often in
parallel, but usually rely on heuristic dispatch and do not provide a
clear, loss-based mechanism for selecting specialists per segment.

In contrast, the architecture I present here performs \emph{task-level}
routing: it computes language-model loss to select relevant experts,
splits the query into segments such as ``solve'', ``explain'' and
``code'', and assigns each segment to the best specialist. The experts
then run in a simple specialist pipeline.

\section{Proposed Architecture}
My system consists of a pool of specialist models and a two-stage
router. Stage~1 selects a subset of relevant experts for the full query.
Stage~2 performs finer-grained routing over individual segments such as
solve, explain, and code.

\subsection{Stage~1: Loss-Based Expert Selection}
Let $N$ be the number of available experts. For a given input query, I
compute the language-model loss of each expert on the full query
prompt. Experts whose loss lies within a fixed threshold of the minimum
are retained as \emph{relevant experts}. This filters out clearly
irrelevant experts while keeping a small candidate set.

Figure~\ref{fig:loss-pick} illustrates this stage. The top row shows the
full pool of experts $E_1,\dots,E_N$. The router computes loss for each
expert and selects those that appear to understand the query best.

\begin{figure}[!t]
    \centering
    % Use absolute path or move the file next to this .tex and use just {Losspicking.png}
    \includegraphics[width=\linewidth]{C:/Users/super/Desktop/MoE_LLM/Pretrained/Losspicking.png}
    \caption{Stage~1 loss-based expert selection. All $N$ experts are
    evaluated on the full query, and only the experts with relatively
    low loss are kept as the relevant expert set for further routing.}
    \label{fig:loss-pick}
\end{figure}

\subsection{Stage~2: Specialist Pipeline After Routing}
After Stage~1, only the relevant experts are considered. The query is
split into semantic segments such as:
\begin{itemize}
    \item a \emph{solve} segment describing the core mathematical task,
    \item an \emph{explain} segment asking for a step-by-step
          explanation,
    \item a \emph{code} segment requesting an implementation.
\end{itemize}
For each segment, the router computes loss using the relevant experts
and assigns the segment to the specialist with the lowest loss on that
segment.

The selected experts are then executed in a simple pipeline:
the solve expert answers the main question, the explain expert builds a
clear explanation (possibly conditioned on the answer), and the code
expert writes the requested implementation. Figure~\ref{fig:pipeline}
shows the conceptual pipeline after routing.

\begin{figure}[!t]
    \centering
    % Use absolute path or move the file next to this .tex and use just {pipeline.png}
    \includegraphics[width=\linewidth]{C:/Users/super/Desktop/MoE_LLM/Pretrained/pipeline.png}
    \caption{Execution pipeline after expert selection. Each segment of
    the query is routed to the most suitable specialist, and the segment
    outputs are combined into a final answer.}
    \label{fig:pipeline}
\end{figure}

\subsection{Training Compute Considerations}
Training several small specialists can be significantly cheaper than
training a single large dense model. Each specialist is trained only on
its own domain (e.g., math, code, or general instruction), instead of
learning everything at once. In practice this means fewer total
effective tokens per model and simpler gradients for each domain,
reducing overall training time while preserving strong performance on
the targeted tasks.

\section{Results}
To evaluate the architecture, I compared the specialist MoE router
against a stronger dense model on a deliberately challenging benchmark
of 50 PhD-level questions in advanced mathematics and theoretical or
computational computer science. The benchmark contains:
\begin{itemize}
    \item 25 questions in real and functional analysis, PDEs,
          probability, topology, approximation theory, and numerical
          analysis;
    \item 25 questions in algorithms, data structures, systems, parallel
          computing, compiler design, and statistical computing in~R.
\end{itemize}

\subsection{Models Compared}
I compare two systems:

\begin{itemize}
    \item \textbf{Model A:} Qwen2.5-VL-3B-Instruct, a 3B generalist
    instruction-tuned model.
    \item \textbf{Model B:} MoE Router, a routed system that dispatches
    query segments among three smaller specialists:
    \emph{Qwen2.5-0.5B-Instruct} (general instruction model),
    \emph{Qwen2.5-0.5B-Coder} (code specialist), and
    \emph{Qwen2.5-1.5B-Math} (mathematics specialist).
\end{itemize}

For each question, both models are prompted once. I grade the answers
manually along three axes on a 0--10 scale:
\begin{itemize}
    \item \emph{Solve}: did the model substantially solve the main
    mathematical or algorithmic task?
    \item \emph{Explain}: was the reasoning clear and logically
    structured?
    \item \emph{Code}: when code was requested, did the model produce
    relevant and plausible code?
\end{itemize}

The grading rubric is:
0--2 = almost no progress, 3--4 = partial or badly broken solution,
5--6 = captures the main ideas but with non-trivial gaps, 7--8 = mostly
correct and fixable by an expert, 9--10 = near-expert performance
(extremely rare on this benchmark).

\subsection{Average Scores}
Averaging over all 50 questions yields the scores summarized in
Table~\ref{tab:avg-scores}.

\begin{table}[!t]
    \centering
    \caption{Average scores over 50 PhD-level questions.}
    \label{tab:avg-scores}
    \begin{tabular}{lccc}
        \toprule
        Model & Solve & Explain & Code \\
        \midrule
        Qwen2.5-VL-3B-Instruct & 5.62 & 5.62 & 4.64 \\
        MoE Router             & 5.14 & 5.14 & 3.70 \\
        \bottomrule
    \end{tabular}
\end{table}

Several observations follow from these results. On \emph{Solve} and
\emph{Explain}, the MoE Router is only about 0.5 points behind the
larger 3B generalist on a 0--10 scale, corresponding to a gap of roughly
10\%. On \emph{Code}, the gap is larger (4.64 vs.\ 3.70 on average),
mainly because the code specialist sometimes produces partial or
underdeveloped implementations rather than fully finished solutions.

Overall, despite using smaller models and a still-simple routing
strategy, the specialist MoE architecture operates in a performance
regime reasonably close to a much larger dense generalist on extremely
difficult problems.

\section{Conclusion}
In this paper I presented a sequential specialist MoE routing
architecture for complex multi-stage queries. Instead of relying on a
single large generalist model, the system uses several small specialists
coordinated by a two-stage loss-based router. Stage~1 selects a subset
of relevant experts for the full query, while Stage~2 assigns individual
query segments such as solve, explain, and code to the most suitable
specialist. The experts then run in a simple pipeline.

On a 50-question PhD-level benchmark in mathematics and computer
science, the specialist MoE system comes within about 0.5 points (on a
0--10 scale) of a 3B dense generalist on solving and explanation, while
lagging somewhat more on code quality. Given its lower training cost and
its modularity, I believe this architecture provides a promising
direction for building multi-domain LLM systems from small, reusable
components.

\begin{thebibliography}{00}

\bibitem{ropmura}
S. J. Wang, H. Liu, and X. Li,
``RopMura: Router-and-Planner-Based Multi-Hop Question Answering,''
in \emph{Proceedings of the 2023 Conference on Empirical Methods in Natural Language Processing (EMNLP)}, 2023.

\bibitem{emrc}
J. Zhang, M. Qi, and L. Huang,
``EMRC: Expert Model Routing for Clinical Question Answering,''
in \emph{Findings of the Association for Computational Linguistics (ACL Findings)}, 2024.

\bibitem{moa}
Y. Du et al.,
``Mixture-of-Agents: Learning to Share Behaviors for Multi-Agent Coordination,''
in \emph{Advances in Neural Information Processing Systems (NeurIPS)}, 2023.

\bibitem{chainofexperts}
A. Kamradt and Y. Tay,
``Chain-of-Experts: Sparse Routing in Transformer Layers,''
arXiv:2403.01852, 2024.

\bibitem{cascaded_llm}
H. Zhou, L. Yang, and Z. Fei,
``Cascaded Large Language Models for Efficient Inference,''
in \emph{Proceedings of the 2023 ACL Workshop on Efficient NLP}, 2023.

\bibitem{frugalgpt}
S. Chen et al.,
``FrugalGPT: How to Use Large Language Models While Reducing Cost and Improving Accuracy,''
arXiv:2305.10333, 2023.

\bibitem{art}
A. Press et al.,
``Ask, Refine, Trust: Causal Decomposition
Improves LLM Reasoning,''
in \emph{International Conference on Learning Representations (ICLR)}, 2024.

\bibitem{leasttomost}
J. Zhou, T. He, and K. Cho,
``Least-to-Most Prompting Enables Complex Reasoning in Large Language Models,''
arXiv:2205.10625, 2022.

\bibitem{llmrank}
J. Liang et al.,
``LLMRank: Ranking, Routing, and Merging Specialized LLMs,''
arXiv:2406.04248, 2024.

\end{thebibliography}

\end{document}
